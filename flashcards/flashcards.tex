%%%%%%%%%%%%%%%%%%%%%%%%%%%%%%%%%%%%%%%%%%%%%%%%%%%%%%%%%%%%%%%%%%%%%%%%%%%%%%%%
%% Fred Brockstedt
% Lernkarten fuer Stochastik I 
% HU-Berlin 2013

\documentclass[avery5371,grid,frame]{flashcards} %% avery5371, avery5388, 

%%%%%%%%%%%%%%%%%%%%%%%%%%%%%%%%%%%%%%%%%%%%%%%%%%%%%%%%%%%%%%%%%%%%%%%%%%%%%%%%
%% Usepackeges
\usepackage[utf8]{inputenc}
\usepackage[german]{babel}
\usepackage{amsmath}
\usepackage{amsfonts}
\usepackage{amssymb}
\usepackage{amsthm}
\usepackage{makeidx}
\usepackage{graphicx}
\usepackage{color}
\usepackage{verbatim}
\usepackage{varioref}
\usepackage{float}
\usepackage{amstext}
\usepackage[unicode=true,
 bookmarks=true,bookmarksnumbered=false,bookmarksopen=false,
 breaklinks=true,pdfborder={0 0 0},backref=page,colorlinks=true]
 {hyperref}
\usepackage{bbold}

%%%%%%%%%%%%%%%%%%%%%%%%%%%%%%%%%%%%%%%%%%%%%%%%%%%%%%%%%%%%%%%%%%%%%%%%%%%%%%%%
%% newcommands
\newcommand{\E}{\mathbb{E}}
\newcommand{\N}{\mathbb{N}}
\newcommand{\R}{\mathbb{R}}

%%%%%%%%%%%%%%%%%%%%%%%%%%%%%%%%%%%%%%%%%%%%%%%%%%%%%%%%%%%%%%%%%%%%%%%%%%%%%%%%
%% flashcard
\cardfrontstyle[\large\slshape]{headings}
\cardbackstyle{empty}

%%%%%%%%%%%%%%%%%%%%%%%%%%%%%%%%%%%%%%%%%%%%%%%%%%%%%%%%%%%%%%%%%%%%%%%%%%%%%%%%
%% Document
\begin{document}

\cardfrontfoot{Stochstik I}

%%%%%%%%%%%%%%%%%%%%%%%%%%%%%%%%%%%%%%%%%%%%%%%%%%%%%%%%%%%%%%%%%%%%%%%%%%%%%%%%
%% Definitionen
\begin{flashcard}[Definition]{Ereigniss}
  Eine Teilmenge $A \in \Omega$ heißt ein \underline{\textit{Ereignis}}.\\\\
Wir sagen, ein Ereignis tritt ein, falls für das realisierte Elementarereignis $\omega \in \Omega$ gilt: $\omega \in A$.\\\\
i) unmögliches Ereignis: $A=\emptyset$\\\\
ii) sicheres Ereignis: $A=\Omega$\\\\
iii) $A$ tritt nicht ein $\Leftrightarrow$ $A^c=\Omega\setminus A$ tritt ein \\\\
\end{flashcard}

\begin{flashcard}[Definition]{$\sigma$-Algebra}
  $\mathcal{A}\subset \mathcal{P}(\Omega)=$Potenzmenge von $\Omega$ heißt $\sigma$-Algebra, falls gilt:\\\\
i) $\emptyset \in \mathcal{A}$\\\\
ii) $A\in \mathcal{A} \Rightarrow A^c \in \mathcal{A}$\\\\
iii) Sind $(A_i)_{i \in \mathbb{N}} \in \mathcal{A}$, so auch $\bigcup\limits_{i=1}^\infty A_i \in \mathcal{A}$\\\\\\
\end{flashcard}


\begin{flashcard}[Definition]{Verteilungsfunktion}
  Die Funktion $F:\R \rightarrow [0,1]$ heißt \index{Verteilungsfunktion}Verteilungsfunktion zu $X$ bzw. $\mu$ und ist definiert durch : $F(b):=\mu((-\infty,b])=P(X\leq b)$ für $b \in \R$.\\\\
\end{flashcard}

\begin{flashcard}{unabhängige ZVen}
  Eine Familie $\left(X_{i}\right)_{i\in I}$ von Zufallvariabeln auf
$\left(\Omega,\mathcal{A},\mathbb{P}\right)$ heißt unabhängig, falls
die von den Zufallsvariablen erzeugten Familie von $\sigma$-Algebren
$\left(\sigma\left(X_{i}\right)\right)_{i\in I}\ \left(\sigma\left(X_{i}\right)=\left\{ \left\{ x\in A|A\in\mathcal{B}\left(\overline{\mathbb{R}}\right)\right\} \right\} \right)
$
unabhängig ist.
\end{flashcard}

\begin{flashcard}[Definition]{Verteilung}
  Seien $X_{1}\ldots X_{n}$ reelwertige Zufallsvariablen auf $\left(\Omega,\mathcal{A},P\right)$.
Dann heißt die Verteilung 
\[
\bar{\mu}:=P\circ\bar{X}^{-1}
\]
von $\bar{X}\left(\omega\right)=\left(X_{1}\left(\omega\right),\ldots,X_{n}\left(\omega\right)\right)$
die \emph{gemeinsame Verteilung}\index{gemeinsame Verteilung} der
$X_{1}\ldots X_{n}$.
\end{flashcard}

\begin{flashcard}[Definition]{Translation und Faltung}
  Sei $x\in\mathbb{R}$. Dann heißt 
\begin{eqnarray*}
T_{x}:\mathbb{R} & \to & \mathbb{R}\\
y & \mapsto & x+y
\end{eqnarray*}
die \index{Translation}\emph{Translation} (um $x$). Für W'Maße $\mu_{1},\mu_{2}$
aud $\left(\mathbb{R},\mathcal{B}\left(\mathbb{R}\right)\right)$
heißt 
\[
\left(\mu_{1}*\mu_{2}\right)\left(A\right)=\int_{\mathbb{R}}\left(\mu_{2}\circ T_{x}^{-1}\right)\left(A\right)\mu_{1}\left(dx_{1}\right)
\]
die \emph{Faltung}\index{Faltung} von $\mu_{1}$ und $\mu_{2}$.
\end{flashcard}
%%%%%%%%%%%%%%%%%%%%%%%%%%%%%%%%%%%%%%%%%%%%%%%%%%%%%%%%%%%%%%%%%%%%%%%%%%%%%%%%
%% Saetze und Lemmata

\begin{flashcard}[Satz]{$\cap$-stabil}
  Seien $\left(B_{i}\right)_{i\in I}$ durchschnitts-stabile unabhängige
Mengensysteme. Dann gilt
\begin{description}
\item [{(i)}] Die $\sigma$-Algebren $\sigma\left(B_{i}\right)$ mit $i\in I$
sind unabhängig.
\item [{(ii)}] Sind $J_{k}$ $k\in\mathcal{K}$ disjunkte (nicht notwendig
endliche) Teilmengen von $I$, so sind
$\sigma\left(\bigcup_{l\in J_{k}}B_{l}\right)\ \left(k\in\mathcal{K}\right)$
unabhängig. \label{s1.2(ii)}
\end{description}
\end{flashcard}

\begin{flashcard}[Satz]{0-1 Gesetz von Kolmogorov}
  Für alle $A\in B_{\infty}$ gilt
\[
P\left(A\right)\in\left\{ 0,1\right\} 
\]
(kein Zufall mehr auf $B_{\infty}$)
\end{flashcard}

\begin{flashcard}{Erwartungswert unabhängiger ZVen}
Seien $X_{1},X_{2},\ldots\geq0$ unabhängig. Dann 
\[
\mathbb{E}\left[X_{1}\ldots X_{n}\right]=\prod_{i=1}^{n}\mathbb{E}\left[X_{i}\right]
\]
\end{flashcard}

\begin{flashcard}[Satz]{Etanadi}
  Seien $\left(X_{i}\right)_{i\in\mathbb{N}}$ \emph{paarweise }unabhängig
in $\mathcal{L}^{1}$, identisch verteilt. Dann gilt \ref{eq:kolmogorov}
\[
\frac{1}{n}\sum_{i=1}^{n}X_{i}\to m\ P-f.s.
\]

\end{flashcard}

\begin{flashcard}[Satz]{Kolmogorov schwache Konvergenz}
Für P-fast alle $\omega\in\Omega$ gilt (im Rahmen des Gesetzes von
Kolmogorov)
\[
\rho_{n}\left(\omega,\bullet\right)\overset{w}{\longrightarrow}\mu
\]
\end{flashcard}

\begin{flashcard}[Satz]{Fubini}
  Sei $f:\mathbb{R}^{n}\to\mathbb{R}$ nicht-negativ oder $\bar{\mu}:=\bigotimes_{i=1}^{n}\mu_{i}$
integrierbar so gilt jede Permutation $i_{1},\ldots,i_{n}$ von $\left\{ 1\ldots n\right\} $
$
\int_{\mathbb{R}^{n}}f\left(x_{1},\ldots,x_{n}\right)\bar{\mu}\left(dx_{1},\ldots,dx_{n}\right)=\int_{\mathbb{R}}\left(\ldots\left(\int_{\mathbb{R}}f\left(x_{1},\ldots,x_{n}\right)\mu_{i_{1}}\left(dx_{i_{1}}\right)\right)\ldots\right)\mu_{i_{n}}\left(dx_{i_{n}}\right)
$
\end{flashcard}

\begin{flashcard}[Satz]{Eindeutigkeitssatz}
  Sei $\mu$ ein Wahrscheinlichkeitsmaß mit charakteristischer Funktion
$\varphi$ (Fourier-Transformierte). Sei $\mu\left(\left\{ a\right\} \right)=\mu\left(\left\{ b\right\} \right)=0$.
Dann gilt
\[
\mu\left((a,b]\right)=\lim_{T\to\infty}\frac{1}{2\pi}\int_{-T}^{T}\frac{e^{-ita}-e^{-itb}}{it}\varphi\left(t\right)dt
\]
Insbesondere legt $\varphi$ das Maß $\mu$ eindeutig fest. 
\end{flashcard}


%%%%%%%%%%%%%%%%%%%%%%%%%%%%%%%%%%%%%%%%%%%%%%%%%%%%%%%%%%%%%%%%%%%%%%%%%%%%%%%%
%% wichite Bemerkungen

\end{document}
